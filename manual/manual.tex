\documentclass[10pt, a4paper]{article}

% include stuff
\usepackage[utf8]{inputenc}
\usepackage{todonotes}
\usepackage{hyperref}
\usepackage{amssymb}
\usepackage{longtable}
%\usepackage{multirow}
\usepackage[T1]{fontenc}

% document related
\title{JACUSA manual \\ Version 1.2}
\author{Michael Piechotta \\ michael.piechotta@age.mpg.de}
\date{March 6, 2017} 

% --------------------------------------------------------------------------------------------------
\begin{document}
% --------------------------------------------------------------------------------------------------
\maketitle \tableofcontents
% --------------------------------------------------------------------------------------------------
\section{Introduction}
JAVA framework for accurate SNV assessment (JACUSA) is a one-stop solution to detect single
nucleotide variants (SNVs) from comparing matched sequencing samples. Robust identification has
proven to be a daunting task due to artefacts specific for NGS-data and employed mapping
strategies. We implement various feature filters that reduce the number of false positives. JACUSA
employs a window-based approach to traverse provided BAM files featuring highly parallel processing.
JACUSA has been extensively evaluated and optimized to identify RNA editing sites in RNA-DNA and
RNA-RNA sequencing samples. JACUSA requires an operating JAVA environment and uses sorted and
indexed BAM files as input.
% --------------------------------------------------------------------------------------------------
\section{Download}
Download the latest version of JACUSA from \href{https://github.com/dieterich-lab/JACUSA/blob/master/build/JACUSA_v1.2.0.jar}. 
Check the source in the repository on \href{https://github.com/dieterich-lab/JACUSA}{GiTHub}.
% --------------------------------------------------------------------------------------------------
\subsection{Installation and requirements}
JACUSA does not need any configuration but needs a correctly configured Java environment.
We developed and tested JACUSA with Java v1.7. If you encounter any Java related problems please
consider to change to Java v1.7.
% --------------------------------------------------------------------------------------------------
\subsection{Sample \textit{in silico} data}
Download sample data that we used for the development of JACUSA. You can choose between different
setups and species where the later greatly influences the data size and running time to detect
variants. The gDNA VS cDNA represents the typical data setup that is encountered in detection of RNA
editing sites via comparing genomic and transcriptomic sequencing reads. In this setup, variants
have been only imputed to the cDNA BAM file. The cDNA VS cDNA data setup can be interpreted as
representing allele specific expression of single variants or differential RNA editing. In this
setup, variants with pairwise different base frequency have been imputed into both cDNA BAM files.
Additionally, to make the identification of variants more challenging SNPs with pairwise similar base
frequencies have been included to both BAM files. This sites should not be identified as true
positive sites.
gDNA data has been simulate with
art\footnote{\href{http://www.niehs.nih.gov/research/resources/software/biostatistics/art/}{art}}
and cDNA reads have been simulated with
flux\footnote{\href{http://sammeth.net/confluence/display/SIM/Home}{flux simulator}}. Read
simulations have been restricted to the corresponding first chromosome of the respective species.
Sample data is available for \textit{C. elegans} ce10 and \textit{Homo sapien} hg19. Each archive
consists of:
\begin{description}
  \item[gDNA.bam, cDNA.bam] BAM files: gDNA.bam and cDNA OR cDNA\_1.bam and cDNA\_2.bam
  \item[snps.txt] Only available for cDNA VS cDNA. Coordinates of imputed SNPs. In both
  BAM files matching SNPs have the same target frequency but different effective or sampled
  frequency. The shape parameter determines how much the sampled frequency will deviate from the
  target frequency in each BAM file. The suffixes: \_cdna\_1 and \_cdna\_2 correspond to the
  respective BAM file
  \item[variants.txt] Coordinates of imputed variants and their target and sample
  frequencies
\end{description}
Available sample data organized by data type and species:
\begin{itemize}
  \item
  \href{http://www.age.mpg.com/software/jacusa/sample_data/hg19_chr1_gDNA_VS_cDNA.tar.gz}{hg19\_chr1\_gDNA\_VS\_cDNA.tar.gz}
  \item
  \href{http://www.age.mpg.com/software/jacusa/sample_data/hg19_chr1_cDNA_VS_cDNA.tar.gz}{hg19\_chr1\_cDNA\_VS\_cDNA.tar.gz}
\end{itemize}
%\todo{This data has to be generated}
%\begin{itemize}
%  \item 
%  \href{http://www.age.mpg.com/software/jacusa/sample_data/ce10_chrI_gDNA_VS_cDNA.tar.gz}{ce10\_chrI\_gDNA\_VS\_cDNA.tar.gz}
%  \item 
%  \href{http://www.age.mpg.com/software/jacusa/sample_data/ce10_chrI_cDNA_VS_cDNA.tar.gz}{ce10\_chrI\_cDNA\_VS\_cDNA.tar.gz}
%\end{itemize}
%--------------------------------------------------------------------------------------------------
\section{Input}
\subsection{Alignment files}
JACUSA needs sorted and indexed BAM files. BAM is a standardized file format for
efficient storage of alignments. Check the manuals for
\footnote{\href{http://samtools.sourceforge.net/}{SAMtools/BCFtools}} and/or
\footnote{\href{http://broadinstitute.github.io/picard/}{picard tools}} for how to use the
respective tool to convert your alignment files to valid JACUSA input BAM.

In the following, commands for SAMtools are presented:
\begin{description}
\item[SAM $\rightarrow BAM$] \begin{verbatim} samtools view -Sb mapping.sam > mapping.bam \end{verbatim}
\item[sort BAM] \begin{verbatim} samtools sort mapping.bam mapping.sorted \end{verbatim} 
\item[index BAM] \begin{verbatim} samtools index mapping.sorted.bam \end{verbatim}
\end{description}

It is a recommended pre-processing step to remove duplicate reads when identifying variants.
Duplicated reads occur mostly due to PCR-artefacts. 
They are likely to harbour false variants and most statistical test require that reads are sampled independently.  
In the following, commands for picarf tools are presented:
\begin{verbatim} java -jar MarkDuplicates.jar \ 
  I=mapping.sorted.bam O=dedup_mapping.sorted.bam \ 
  M=duplication.info
\end{verbatim}
Invoke JACUSA with the additional command line option ``-F 1024'' to filter read that have been marked as duplicates.

\subsubsection{Strand information}
In v1.2, support for stranded paired end reads has been added to JACUSA. Depending on the employed sequencing library, 
JACUSA can use the strand orientation to build pileups.

\begin{center}
\emph{WARNING} \\
Since v1.2 the format of ``-P'' has changed!
The format has been inspired by tophat's \href{http://ccb.jhu.edu/software/tophat/manual.shtml} library type parameter. 
\end{center}

With the command line parameter ``-P,--build-pileup <BUILD-PILEUP>'' the user can choose from combinations of:
\begin{description} 
\item[FR-FIRSTSTRAND] STRANDED library - first strand sequenced,
\item[FR-SECONDSTRAND] STRANDED library - second strand sequenced, and
\item[UNSTRANDED] UNSTRANDED library.
\end{description}
to define if strand information will be utilized to build pileups.

In order to identify RNA editing sites by comparing gDNA and \emph{stranded} RNA-Seq (single or paired end) use:
\begin{description} 
\item[first strand sequenced] ``-P UNSTRANDED,FR-FIRSTSTRAND''
\item[second strand sequenced] ``-P UNSTRANDED,FR-SECONDSTRAND''
\end{description}.
When your RNA-Seq is unstranded use: ``-P UNSTRANDED,UNSTRANDED'' and infer the correct orientation from annnotation.
%--------------------------------------------------------------------------------------------------
\subsection{Traverse BED-like file}
Variant detection can be limited to specific regions of the genome or transcriptome.  Provide a
minimalistic BED-like file to restrict the search to this region(s) or site(s). Remaining region(s)
of the BAM files will not be considered.

In the following traverse file, the search is confined to a 100nt region on contig 1
staring at 1,000 and a single site on contig 2 at coordinates 10,000:
\begin{table}
\centering
\caption{Example of BED-like traverse file}
\label{tb:traverse_file}
\begin{tabular}{lll}
\textbf{contig} & \textbf{start} & \textbf{end} \\
\hline
1 & 1000 & 1100 \\
2 & 10000 & 10000 \\
\multicolumn{3}{c}{}
\end{tabular}
\end{table}
HINT: Many individual sites will slow down JACUSA. If possible, try to merge nearby sites into
contiguous regions and extract specific sites from JACUSA output with
bedtools\footnote{\href{http://bedtools.readthedocs.org/en/latest/}{bedtools}} intersect:
\begin{description}
\item[merge sites] \begin{verbatim} 

bedtools merge -d 500 singular_sites.bed > \ 
  contigous_regions.bed
\end{verbatim}
\item[run JACUSA] \begin{verbatim} 

java -jar JACUSA.jar call-2 -b contigous_regions.bed -r
JACUSA.out mapping_1.sorted.bam mapping_2.sorted.bam
\end{verbatim}
\item[extract sites] \begin{verbatim}

bedtools intersect -wa -a JACUSA.out -b singular_sites.bed
\end{verbatim}
\end{description}
%--------------------------------------------------------------------------------------------------
\section{Output}
JACUSA writes its output to a user defined file or a pipe. When using multiple threads, JACUSA will
create a gzipped temporary file for each allocated thread. Chosen command line parameters
and current genomic position are printed to the command prompt. Furthermore, depending on
the provided command line parameters, JACUSA will generate a file with sites that have been
identified as potential artefacts when ``-s" is provided. Currently, JACUSA supports the following
output formats, controlled by ``-f'':
\begin{itemize}
  \item Default (JACUSA output)
  \item Variant Call Format
  (VCF)\footnote{\href{http://samtools.github.io/hts-specs/VCFv4.1.pdf]}{VCF file format}}
\end{itemize}
The default output format is based on
BED6\footnote{\href{http://genome.ucsc.edu/FAQ/FAQformat.html\#format1}{BED file format}} with
additional JACUSA specific columns. The actual number of columns depends on number of provided BAM
files.
\begin{table}[ht]
\caption{JACUSA default output format}
{\small
\begin{tabular}{lcccccc|cccc}
%contig & start & end & name & score & strand & bases11 & bases22 & info & filter\_info \\
Column: & 1 & 2 & 3 & 4 & 5 & 6 & 7 & 8 & 9 & 10 \\
\hline
& 1 & 100 & 101 & variant & $8.07\ldots$ & - & 0,0,0,6 & 0,6,0,0 & * & * \\	
& \multicolumn{6}{c|}{\ldots} & \multicolumn{4}{c}{\ldots}
\end{tabular}}
\end{table}
\begin{description}
\item[(1, 2, 3) contig + start + end] 0-based, genomic coordinates of potential variant site
\item[(4) name] Currently, constant string: ``variant''. This dummy field is to ensure BED6
compatibility
\item[(5) score] Test-statistic $z \in \mathbb{R}$ that indicates the likelihood that this is a true
variant. Higher number indicates a higher likelihood for a variant
\item[(6) strand] Possible values are: ``.'', ``+'', and ``-'' which correspond to ``unstranded'',
``positive strand'', and ``negative strand'' respectively. If strand is != ``.'', then the following base columns
will be indicating base counts according to the strand - inverted base count if on the ``negative
strand''
\item[(7,8) basesIJ] The number of base columns depends on the number of BAM files. In basesIJ: $I$
corresponds to sample and $J$ to the respective replicate. Numbers indicate the base count of the
following base vector: $(A, C, G, T)$
\item[(9) info] Additional info for this specific site. Currently, details about the parameter
estimation of the Dirichlet-Multinomial can be shown. If nothing provided, the empty field is equal
to ``*''
\item[(10) filter\_info] Relevant, if feature filter(s) $X$ have been provided with ``-a X'' on the
command line. The column will contain a comma-separated list of feature filters that predict this
site to be a potential artefact. Possible values are: \\ 
\begin{tabular}{lp{.8\textwidth}}
\textbf{Value} & \textbf{Description of potential artefact} \\
\hline
D & Variant call in the vicinity of Read Start/End, Intron, and/or INDEL position \\
B & Variant call in the vicinity of Read Start/End \\
I & Variant call in the vicinity of INDEL position \\
S & Variant call in the vicinity of Splice Site \\
Y & Variant call in the vicinity of homopolymer \\
M & Max allowed alleles exceeded \\ 
H & ``Control" sample contains non-homozygous pileup \\
d & Some pileup exceeds max depth
\end{tabular}
\end{description}
%--------------------------------------------------------------------------------------------------
\section{Usage}
Calling JACUSA without any arguments will print the available tools which currently are:
\begin{verbatim}
java -jarJACUSA.jar
	call-1  Call variants - one sample
	call-2  Call variants - two samples
  pileup  SAMtools like mpileup for two samples
\end{verbatim}
{\small
\begin{verbatim}
usage: jacusa.jar [OPTIONS] BAM1_1[,BAM1_2,BAM1_3,...] BAM2_1[,BAM2_2,BAM2_3,...]
\end{verbatim}}
\begin{center}
{\small
\begin{longtable}{p{.4\textwidth}p{.6\textwidth}}
 -a,--pileup-filter <PILEUP-FILTER>             & chain of PILEUP-FILTER to apply to pileups: \\
                                                & B | Filter distance to Read Start/End. Default: 6:0.5 (F:distance:min\_ratio) \\
                                                & R | Rare event filter. Default: 1:2:0.1 (R:pool:reads:level) \\
                                                & S | Filter distance to Splice Site. Default: 6:0.5 (S:distance:min\_ratio) \\
                                                & D | Filter distance to Read Start/End, Intron, and INDEL position. Default: 5:0.5 (D:distance:min\_ratio) \\
                                                & H | Filter non-homozygous pileup/BAM in sample 1 or 2 (MUST be set to H:1 or H:2). Default: none \\
                                                & I | Filter distance to INDEL position. Default: 6:0.5 (I:distance:min\_ratio) \\
                                                & Y | Filter wrong variant calls in the vicinity of homopolymers. Default: 7 (Y:length) \\
                                                & L | Min difference filter. Default: 0:0:0.1 (R:pool:reads:level) \\
                                                & M | Max allowed alleles per parallel pileup. Default: 2 \\
                                                & Separate multiple PILEUP-FILTER with ',' (e.g.: D,I) \\
 -b,--bed <BED>									                & BED file to scan for variants \\
 -C,--base-config <BASE-CONFIG> 				        & Choose what bases should be considered for variant calling: TC or AG or ACGT or AT$\ldots$. Default: ACGT \\
 -c,--min-coverage <MIN-COVERAGE>               & filter positions with coverage < MIN-COVERAGE. Default: 5 \\
 -c1,--min-coverage1 <MIN-COVERAGE1>            & filter 1 positions with coverage < MIN-COVERAGE1. Default: 5 \\
 -c2,--min-coverage2 <MIN-COVERAGE2>            & filter 2 positions with coverage < MIN-COVERAGE2. Default: 5 \\
 %-D,--debug                                     Enable debug modus
 -d,--max-depth <MAX-DEPTH>                     & max per-BAM depth. Default: -1 \\
 -d1,--max-depth1 <MAX-DEPTH1>                  & max per-sample 1 depth. Default: -1 \\
 -d2,--max-depth2 <MAX-DEPTH2>                  & max per-sample 2 depth. Default: -1 \\
 -F,--filter-flags <FILTER-FLAGS>               & filter reads with flags FILTER-FLAGS. Default: 0 \\
 -f,--output-format <OUTPUT-FORMAT>             & Choose output format: \\
                                                & <*> D: Default JACUSA \\
                                                & < > V: VCF Output format. Option -P will be ignored (VCF is unstranded) \\
    --filterNH\_1 <NH-VALUE>                    & Max NH-VALUE for SAM tag NH \\
    --filterNH\_2 <NH-VALUE>                    & Max NH-VALUE for SAM tag NH \\
    --filterNM\_1 <NM-VALUE>                    & Max NM-VALUE for SAM tag NM \\
    --filterNM\_2 <NM-VALUE>                    & Max NM-VALUE for SAM tag NM \\
 -h,--help                                      & Print usage information \\
 -i1,--invert-strand1                           & Invert strand of 1 sample. Default false \\
 -i2,--invert-strand2                           & Invert strand of 2 sample. Default false \\
 -m,--min-mapq <MIN-MAPQ>                       & filter positions with MAPQ < MIN-MAPQ. Default: 20 \\
 -m1,--min-mapq1 <MIN-MAPQ1>                    & filter 1 positions with MAPQ < MIN-MAPQ1. Default: 20 \\
 -m2,--min-mapq2 <MIN-MAPQ2>                    & filter 2 positions with MAPQ < MIN-MAPQ2. Default: 20 \\
 -P,--build-pileup <BUILD-PILEUP>               & Choose the library types and how parallel pileups are build for sample1(s1) and sample2(s2). \\
                                                & Format: s1,s2. \\
                                                & Possible values for s1 and s2: \\
                                                & FR-FIRSTSTRAND    STRANDED library - first strand sequenced \\
                                                & FR-SECONDSTRAND   STRANDED library - second strand sequenced \\
                                                & UNSTRANDED    UNSTRANDED library \\
                                                & default: UNSTRANDED,UNSTRANDED \\
 -p,--threads <THREADS>                         & use \# THREADS. Default: 1 \\
 -q,--min-basq <MIN-BASQ>                       & filter positions with base quality < MIN-BASQ. Default: 20 \\
 -q1,--min-basq1 <MIN-BASQ1>                    & filter 1 positions with base quality < MIN-BASQ1. Default: 20 \\
 -q2,--min-basq2 <MIN-BASQ2>                    & filter 2 positions with base quality < MIN-BASQ2. Default: 20 \\
 -R,--show-ref <SHOW-REF>                       & Add reference base to output. BAM file(s) must have MD field! \\
 -r,--result-file <RESULT-FILE>                 & results are written to RESULT-FILE or STDOUT if empty \\
 -s,--separate                                  & Put feature-filtered results in to a separate file (= RESULT-FILE.filtered)\\
 -T,--threshold <THRESHOLD>                     & Filter positions depending on test-statistic THRESHOLD default: DO NOT FILTER \\
 %-u,--modus <MODUS>                             & Choose between different modes: \\
 %                                               & <*> DirMult-CE : Compound Err. (estimated err.{0.01} + phred score) \\
 %                                               & < > DirMult-RCE : Robust Compound Err. \\
 -v,--version                                   & Print version information. \\
 -W,--thread-window-size <THREAD-WINDOW-SIZE>   & size of the window used per thread. Default: 100000 \\
 -w,--window-size <WINDOW-SIZE>                 & size of the window used for caching. Make sure this is greater than the read size and smaller than THREAD-WINDOW-SIZE>. Default:
 10000
\end{longtable}}
\end{center}
%--------------------------------------------------------------------------------------------------
\subsection{Single sample mode}
With the introduction of JACUSA v1.2, single sample support(call-1) has been added that allows calling variants against a reference. 
Internally, an \textit{in silico} sample is created from information provided by the MD field in BAM files - see samtools manual for details on MD field.
Single sample mode requires that BAM files contain the MD field. We strongly recommend to use ``samtools calmd'' to add or overwrite existing MD fields in BAM files:
\begin{verbatim}
samtools calmd mapping.sorted.bam reference.fasta > mapping.sorted.MD.bam
\end{verbatim}
After populating the MD field, variants against the reference can be called with:
\begin{verbatim} 
java -jar JACUSA.jar call-1 
-r JACUSA.out mapping.sorted.MD.bam
\end{verbatim}
%--------------------------------------------------------------------------------------------------
\subsection{SAMtools like mpileup for two samples}
See ``Call variant - two samples'' for details.
%--------------------------------------------------------------------------------------------------
\section{Identification of RNA editing sites}
Use the following command line to identify RNA-DNA differences in BAM files that might give rise to RNA editing sites:
\begin{verbatim}
java -jar call-2 -r JACUSA.out -s -a H:1 gDNA.bam cDNA.bam
\end{verbatim}
Option ``-a H:1'' ensures that potential polymorphisms in gDNA will be eliminated as artefacts. The number $x \in \{1, 2\}$
determines which sample has to be homomorph - in this case: gDNA.bam.

Use the following command line to identify RNA-DNA differences:
\begin{verbatim}
java -jar call-2 -r JACUSA.out -s cDNA1.bam cDNA2.bam
\end{verbatim}
WARNING: If you want to identify RNA-RNA differences make sure NOT to use the filter ``-a H:x''! Otherwise, potential valid variants will be filtered out. 

%--------------------------------------------------------------------------------------------------
\section{Used libraries}
picard tools
%\todo{complete list}
%--------------------------------------------------------------------------------------------------
\end{document}
%--------------------------------------------------------------------------------------------------
